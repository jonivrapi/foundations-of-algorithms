\documentclass{article}
\usepackage{a4wide}
\usepackage{amsmath}
\usepackage{url}
\usepackage{amssymb}
\usepackage{clrscode3e}
\usepackage{tikz}
\usepackage{bm}
\usepackage{graphicx}
\graphicspath{ {./images/} }

\title{Homework 7}
\author{Joni Vrapi}
\date{12/5/2022}

\begin{document}

\maketitle

\textbf{Statement of Integrity:} I, Joni Vrapi, attempted to answer each question honestly and to the best of my abilities. I cited any, and all, help that I received in completing this assignment.

\hfill

\textbf{Problem 1.} 
decision variables are the variables you need to make a decision on - eg how many bowls/mugs do you need to produce to have the most revenue

\hfill

\textbf{Problem 2a.} If we let $x = (x_1, x_2, x_3)$ and $y = (y_1, y_2, y_3)$, then the expected loss for Player 1 can be calculated as $\sum{xAy^T}$ resulting in:

\begin{gather}
    x_1(y_2-y_3) + x_2(y_3-y_1) + x_3(y_1-y_2)
\end{gather}

\hfill

\textbf{Problem 2b.} Player 2 playing any strategy other than $(\frac{1}{3}, \frac{1}{3}, \frac{1}{3})$ will allow Player 1 to adjust his strategy to play off of Player 2's strategy, resulting in an expected gain for Player 1. For example, assume Player 2 plays $(1, 0, 0)$. Player 1 would then adjust his strategy to play $(0, 1, 0)$ resulting in an expected gain of 1 (from equation 1). Likewise, it is obvious that regardless of what strategy Player 2 chooses, other than $(\frac{1}{3}, \frac{1}{3}, \frac{1}{3})$, Player 1 will always be able to play off of it towards an expected gain.

\hfill

\textbf{Problem 2c.} 

\hfill

\textbf{Problem 3.} 

\newpage
\bibliography{citation} 
\bibliographystyle{ieeetr}

\end{document} 
