\documentclass{article}
\usepackage{amsmath}
\usepackage{url}
\usepackage{amssymb}
\usepackage{clrscode3e}
\usepackage{a4wide}

\title{Programming Assignment 4}
\author{Joni Vrapi}
\date{12/11/2022}

\begin{document}

\maketitle

\textbf{Statement of Integrity:} I, Joni Vrapi, attempted to answer each question honestly and to the best of my abilities. I cited any, and all, help that I received in completing this assignment.

\hfill

\textbf{Problem 1c.} 

\begin{codebox}
    \Procname{$\proc{Process-Signal}(x, y, s)$}
    \li xSet = ySet = noiseSet = []
    \li xMovingIndex = yMovingIndex = xCompleted = yCompleted = 0
    \li \For index = 0 to length(s) \Do
    \li \If s[index] is the i'th character of both x and y \Then
    \li \If x has been completed more than y \Then
    \li ySet.append(index + 1)
    \li move y index by 1 through y 
    \li \ElseIf xCompleted == yCompleted \Then
    \li \If xMovingIndex is ahead of yMovingIndex \Then
    \li xSet.append(index + 1)
    \li move x index by 1 through x
    \li \Else 
    \li ySet.append(index + 1)
    \li move y index by 1 through y \End
    \li \Else
    \li xSet.append(index + 1)
    \li move x index by 1 through x \End
    \li continue \End
    \li
    \li \If s[index] is the i'th character of x \Then
    \li xSet.append(index + 1)
    \li move x index by 1 through x 
    \li continue \End
    \li
    \li \If s[index] is the i'th character of y \Then
    \li ySet.append(index + 1)
    \li move y index by 1 through y
    \li continue \End
    \li
    \li \If length(noiseSet != 0) \Then
    \li assign to noiseSet an array of integers from 1 to length(s) 
    \li that do not include any numbers that are in xSet or ySet \End
    \li
    \li \If length(xSet) + length(ySet) + length(noiseSet) == length(s) \Then
    \li this is an interweaving \End
\end{codebox}

\hfill

\textbf{Problem 1d.} This algorithm processes the signal "as it comes in" via a single for loop which iterates through the input string only once for an $O(n)$ time. There are many comparisons that are made, all of which are $O(1)$ operations. Finally, to get the noise in the signal, it generates another array of size $m < n$ in $O(m)$ time. In total, this is $O(n) + O(m)$ which is linear with respect to inputs, so this is an $O(n)$ algorithm.

\hfill

\textbf{Problem 2c.} In three out of the four test cases, there was no noise, so per my analysis in Problem 1d, I would expect the length of the input to equal the number of iterations my program made. The instrumentation in tests ${s1, s2, s4}$ confirm this. The third test case $s3$ has 12 elements of noise, on an input of length 33. Per my analysis in Problem 1d, I would expect the total number of iterations to be $33 + 12 = 45$, and the instrumentation also confirms this. Therefore, we can conclude that this algorithm does in fact run in $O(n)$ time. 

\newpage
\bibliography{citation} 
\bibliographystyle{ieeetr}

\end{document} 
